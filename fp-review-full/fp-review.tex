\documentclass{zarticle}
\addbibresource{fp-review.bib}
\definecolor{todocolor}{RGB}{127,0,0}
\def\todo#1{{\color{todocolor}\bfseries [#1]}}
\def\needcite#1{\todo{cite: #1}}
\begin{document}

\title{Unmasking Web Users and Activities:
  A review of the literature on traffic analysis of the modern Internet}
\author{Zachary Weinberg}
\date{\today}
\maketitle

\begin{abstract}
Abstract abstract, abstract abstract abstract.
\end{abstract}

\section{Introduction}

Most off-the-shelf secure network protocols provide some form of the
conceptual \emph{secure channel} for passing messages back and forth
between two endpoints.  A secure channel offers three security
guarantees: \emph{confidentiality}---an attacker cannot observe the
content of messages in transit; \emph{integrity}---an attacker cannot
modify messages in transit; and \emph{authenticity}---an attacker
cannot impersonate either endpoint.  These guarantees collectively
offer a solid foundation for secure application development, but they
are not sufficient for complete security in all situations.  Even with
a secure channel in place, an eavesdropper can easily learn the origin
and destination addresses of each packet, the amount of application
data it carries, its position within a TCP stream, and other
metainformation, such as any TCP options that are present.

Of this information, the origin and destination are most sensitive;
sometimes the information that Alice did at some point talk to Bob is
enough to send Alice to prison!\footnote{As an extreme example, the
  Soviet Union notoriously criminalized “contacts leading to suspicion
  of espionage.”~\needcite{solzhenitsyn?}} A \emph{secure proxy}
relays traffic from Alice to Bob and vice versa, so an eavesdropper at
any single point in the network cannot tell that they are
communicating with each other.  At most, the eavesdropper will know
that one of them is communicating with the proxy.  This is the
additional security guarantee of
\emph{unlinkability}~\cite{pfitzmann2010terminology}.

A secure proxy itself knows both the source and the destination of
each message.  Even if run by people with the best of intentions, it
is vulnerable to offline coercion to expose that
information. \todo{something about anon.penet.fi here}.  A \emph{mix
  network}~\needcite{chaum1978} replaces the single relay with a chain
of relays, each of which knows only the previous and next hop;
subverting any one relay thus reveals no more than eavesdropping would
have. \todo{Explain why three relays are used in practice.}  Some of
the research we review studied the behavior of secure proxies, while
other papers examined mix networks.  For the most part, the
differences between the two are not relevant to our discussion, so we
will refer to both interchangeably as \emph{unlinkable channels}
except where the difference matters.

Unlinkable channels don't generally conceal the size or the timing of
messages.  There are exceptions: Chaum's original mix design (intended
for email) did include substantial delays at each relay in order to
conceal timing, but this is not considered acceptable for modern
interactive protocols such as HTTP: current-generation mix networks,
such as Tor~\cite{dingledine2004tor}, treat their \emph{low latency}
as a valuable feature.  Unlinkable channels may or may not disguise
the TCP session to which each packet belongs.

\subsection{Traffic Analysis}



\nocite{*}\printbibliography
\end{document}
