\section{Threat Models for Fingerprinting}

\begin{figure*}%
\begin{tikzpicture}[every node/.style={align=center}]
\node [draw,shape=circle] (s1) at (4,0.75) {};
\node [draw,shape=circle] (s2) at (4,0) {};
\node [draw,shape=circle] (s3) at (4,-0.75) {};
\node at (5.5,0) {Small, fixed set\\of web servers};
\draw (4.25,1) -- ++(0.05,0) -- ++(0,-2) -- ++(-0.05,0) ;

\coordinate [left=1.5cm of s2] (c) {};

\node [draw, left=1cm of s2, cloud, aspect=2] (cl) { \strut } ;

\node [draw,left=2.5cm of c] (p) {Unlinkability\\relay};
\node [draw,left=1cm of p] (e) {Eavesdropping\\network\\router} ;
\node [draw,left=1cm of e] (v) {Targeted client\\(scripted browser)} ;

\draw (v) -- (e) (e) -- (p) (p) -- (cl) ;

\draw [dotted] (cl.west) -- (c)
      (c) -- (s1)
      (c) -- (s2)
      (c) -- (s3) ;

\end{tikzpicture}%
\caption{Laboratory topology for fingerprinting experiments.}%
\label{f:labtopology}%
\end{figure*}

Nearly all of the existing research on website fingerprinting has been
carried out in what we might call a laboratory environment.
Specifically, regardless of the attacker's goals (within-site,
identify-site, identify-user), experiments have been conducted using
roughly the network topology shown in figure~\ref{f:labtopology}.
Relative to the conditions an attacker “in the field” would likely
encounter, this configuration is simplified in three ways.  The
topology is \emph{fixed} for the course of the experiment; the
targeted client does not move around in the network, and it always
uses the same initial unlinkability relay.  (The latter of these is
more realistic than the former.  Proxy servers tend to have stable IP
addresses, and Tor clients choose a small set of “guard” nodes to use
for all connections, changing them only at long intervals.)

Also, the eavesdropping router sees \emph{only one traffic source},
the targeted client.  This may seem like a valid assumption, as the
attacker may well conduct their eavesdropping one network hop from the
target.  Modern WiFi hubs, DSL modems, and the like are full-fledged
computing platforms, running poorly-secured management software, which
could be exploited to install a monitor---if it isn't already present
in the factory install; lately telcos are notorious for spying on
their customers.\needcite{???}  However, these devices are installed
for the use of an entire household (1--10 people) or office (10--50
people), or as a public amenity, say at a café (20--100 people at any
given time).  Even if there is only one client \emph{computer} ever
connected via the eavesdropping router, that computer is likely to
generate HTTP traffic which is uninformative regarding the target
\emph{human}: automatic software updates, for instance, are carried
out over HTTP nowadays.  If the eavesdropper is more than one hop from
the target, their situation is even worse: each IP address sending
them traffic could conceal dozens of clients behind a NAT gateway.
(NAT can be detected~\cite{elie2005timestamp,krmivcek2009netflow} but
that doesn't help the adversary decide which packet streams to pay
attention to.)

Finally, and perhaps most significantly, the set of servers that the
target might attempt to contact is \emph{small} and \emph{fixed}.  The
largest study to date~\todo{Sun et al?} considered a universe of the
20,000 most frequently contacted servers as recorded by a departmental
Web proxy.  Contrast the 670,000,000 distinct websites reported in
June 2013 by
Netcraft\footnote{\url{http://news.netcraft.com/archives/2013/06/06/june-2013-web-server-survey-3.html}}
or even the 100,000 sites which are sufficiently popular to have
stable Alexa rankings\footnote{\url{http://www.alexa.com/faqs/?p=134}}.
Furthermore, only studies focusing on the contents of a \emph{single}
server investigate pages other than the “front page” of each server,
Front pages, which express the site's corporate identity, are
plausibly more different from each other than internal pages, which
are often a stock set of “chrome” wrapped around a blob of text.

\subsection{The Adversary's Goals}

To develop more realistic threat models for fingerprinting, we must
first consider the adversary's goals.  \todo{one of the early papers
did discuss this a bit}  
