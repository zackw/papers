\begin{abstract}
Internet censorship by governments is an increasingly common practice
worldwide.  Internet users and censors are locked in an arms race: as
users find ways to evade censorship schemes, the censors develop
countermeasures for the evasion tactics.  One of the most popular and
effective circumvention tools, Tor, must regularly adjust its network
traffic signature to remain usable.

We present StegoTorus, a tool that comprehensively disguises Tor from
protocol analysis.  To foil analysis of packet contents, Tor's traffic
is steganographed to resemble an innocuous cover protocol, such as
HTTP.  To foil analysis at the transport level, the Tor circuit is
distributed over many shorter-lived connections with per-packet
characteristics that mimic cover-protocol traffic.  Our evaluation
demonstrates that StegoTorus improves the resilience of Tor to
fingerprinting attacks and delivers usable performance.

\smallskip\noindent\textbf{Categories and Subject Descriptors:}
C.2.0 [\textbf{Computer-Com\-mu\-ni\-cation Networks}]: Security and protection;
K.4.1 [\textbf{Public Policy Issues}]: Transborder data flow

\smallskip\noindent\textbf{General Terms:} Algorithms, Design, Security

\smallskip\noindent\textbf{Keywords:} Anticensorship, Circumvention Tools, Cryptosystems, Steganography

\end{abstract}
