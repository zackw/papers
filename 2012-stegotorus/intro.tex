\section{Introduction}\label{s:intro}

Freedom of speech and decentralization are bedrock principles of the
modern Internet.  John Gilmore famously said that “the Net interprets
censorship as damage, and routes around it”~\cite{q-gilmore}.  It is
more difficult for a central authority to control what is published on
the Internet than on older, broadcast-based media; in 2011, the
Internet's utility to the “Arab Spring” revolutions prompted a
spokesman for the US Department of State to label it “the Che Guevara
of the 21st century”~\cite{q-ross}.  Nonetheless, national governments
can easily inspect, manipulate, and block nearly all network traffic
that crosses their borders. Over a third of all nations impose
“filters” on their citizens' view of the Internet~\cite{n-global}.  As
the Internet continues to grow in scope and importance, we can expect
that governments will only increase their efforts to control
it~\cite{q-celine}.

Tools for evading online censorship are nearly as old as the
censorship itself~\cite{c-peacefire,c-howto-book}.  At present, one of
the most effective circumvention tools is “Tor”~\cite{c-tor}.  Tor
provides anonymity for its users by interposing three relays between
each user and the sites that the user visits.  Each relay can decrypt
just enough of each packet to learn the next hop. No observer at any
single point in the network, not even a malicious relay, can know both
the source and the destination of Tor traffic.

Although Tor was not designed as an anticensorship tool, it works well
in that role.  Repressive governments respond by blocking Tor
itself. In 2010 and 2011, Iran attempted to block Tor traffic by
scanning TLS handshakes for Diffie-Hellman parameters and/or
certificate features that were characteristic of
Tor~\cite{n-iran1,n-iran2}.  China employed a similar technique but
enhanced it with active probing of the suspected Tor relay, mimicking
the initial sequence of Tor protocol messages in
detail~\cite{n-china-active}.  The Tor developers defeated these
blocks with small adjustments to their software.

For a few days in early 2012, Iran blocked \emph{all} outbound HTTPS
connections to many websites~\cite{n-iran3}, including Tor's primary
site.\footnote{\url{https://www.torproject.org/}} To evade this more
drastic blockade, Tor deployed a program called \textit{obfsproxy}
(for “obfuscating proxy”)~\cite{n-iran-ob}.  Obfsproxy applies an
additional stream cipher to Tor's traffic.  This frustrates any filter
looking for a specific plaintext pattern (such as a TLS handshake),
but does not significantly alter packet sizes and timing.  As we will
discuss in Section~\ref{s:detect-tor}, this means that Tor is still
easily fingerprintable.

\smallskip\noindent\textbf{Contributions:} In this paper, we present
an elaboration on the obfsproxy concept, StegoTorus.  StegoTorus
currently consists of:
\begin{compactitem}
\item A generic architecture for concealing Tor traffic within an
  innocuous “cover protocol” (Section~\ref{s:arch}).
\item A novel encrypted transport protocol geared specifically for the
  needs of steganography (Section~\ref{s:chop}).
\item Two proof-of-concept steganography modules (Section~\ref{s:steg}).
\end{compactitem}
We will demonstrate the ease of detecting un-camouflaged Tor traffic
and StegoTorus' effectiveness at concealing it, even with the current
proof-of-concept steganography (Sections~\ref{s:detect-tor}
and~\ref{s:detect-fb}).  We will also demonstrate that StegoTorus
imposes a reasonable amount of overhead for what it does
(Section~\ref{s:peval}).

We anticipate that censors will adapt quickly to this advance on the
circumvention side of the arms race; more sophisticated and varied
steganography modules are under active development.  Ultimately, an
attacker will need to defeat \emph{all} of the steganography modules
used by StegoTorus to block Tor traffic.
