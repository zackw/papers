\section{Threat Models for Fingerprinting}

\begin{figure*}[t!]%
\begin{tikzpicture}[every node/.style={align=center,draw}]
\matrix [draw=none]
{
   \node (a) {Targeted client};
&[5mm]
   \node (b) {Eavesdropping\\router};
&[10mm]
   \node (c) [cloud,aspect=2] {\strut};
&[7.5mm]
   \foreach \y/\l in {5mm/da,0mm/db,-5mm/dc} \node [circle] (\l) at (0,\y) {};
   \coordinate [above=0.5mm] (d0) at (da.north) ;
   \coordinate [below=0.5mm] (d1) at (dc.south) ;
&[1mm]
   \draw [decorate,decoration={brace,amplitude=2mm}] (d0 -| 0,0) -- (d1 -| 0,0);
&[1mm]
   \node (e) [draw=none] {Fixed, small set\\of web servers};
\\
};
\draw (a) edge (b)
      (b) edge (c)
      (c) edge (da)
      (c) edge (db)
      (c) edge (dc)
;
\end{tikzpicture}%
\caption{Laboratory topology for fingerprinting experiments.}%
\label{f:lab.topology}%
\end{figure*}

\begin{figure*}[t!]%
\begin{tikzpicture}[every node/.style={align=center,draw}]
\matrix[draw=none,column sep=3mm,row sep=2mm]
{
&[4mm]
&[3mm]
&[3mm]
&
  \node (t1) [shape=circle] at (-6mm,6mm) {};
  \node (t2) [shape=circle,draw=none] at (-2mm,2mm) {};
  \node (t3) [shape=circle,draw=none] at (2mm,-2mm) {};
  \node (t4) [shape=circle] at (6mm,-6mm) {};
  \foreach \xy in {(-3mm,3mm),(-1mm,1mm),(1mm,-1mm),(3mm,-3mm)}
    \fill \xy circle (0.75mm) ;
\\
\node (c1) at (0,-8.49mm) {};
\node (c2) [draw=none] at (0,-5.66mm) {};
\node (c3) [draw=none] at (0,-2.83mm) {};
\node (c4) [draw=none] at (0,0) {};
\node (c5) [draw=none] at (0,2.83mm) {};
\node (c6) [draw=none] at (0,5.66mm) {};
\node (c7) at (0,8.49mm) {};
  \foreach \xy in {(0,-4.24mm),(0,-1.41mm),(0,1.41mm),(0,4.24mm)}
    \fill \xy circle (0.666mm) ;
&
  \node (cloud1) [cloud,aspect=2] {\textit{5--10 hops}};
&
  \node (eve) {Backbone router\\with IP blacklist};
&
  \node (cloud2) [cloud,aspect=2] {\textit{5--10 hops}};
&
\\
&
&
&
&
  \node (d1) [shape=circle] at (6mm,6mm) {};
  \node (d2) [shape=circle,draw=none] at (4mm,4mm) {};
  \node (d3) [shape=circle,draw=none] at (2mm,2mm) {};
  \node (d4) [shape=circle,draw=none] at (0mm,0mm) {};
  \node (d5) [shape=circle,draw=none] at (-2mm,-2mm) {};
  \node (d6) [shape=circle,draw=none] at (-4mm,-4mm) {};
  \node (d7) [shape=circle] at (-6mm,-6mm) {};
  \foreach \xy in {(-3mm,-3mm),(-1mm,-1mm),(1mm,1mm),(3mm,3mm)}
    \fill \xy circle (0.75mm) ;
\\
};
\draw (cloud1) edge (eve)
      (eve) edge (cloud2) ;
\foreach \n in {c1,c2,c3,c4,c5,c6,c7} \draw (\n) edge (cloud1) ;
\foreach \n in {t1,t2,t3,t4} \draw (cloud2) edge (\n) ;
\foreach \n in {d1,d2,d3,d4,d5,d6,d7} \draw (cloud2) edge (\n);

\draw [decorate,decoration={brace,amplitude=2mm}]
  ($(c1.south west)+(-1mm,-1mm)$) -- ($(c7.north west)+(-1mm,1mm)$) ;
\draw [decorate,decoration={brace,amplitude=2mm}]
  ($(t1.north)+(0,1mm)$) -- ($(t4.east)+(1mm,0)$) ;
\draw [decorate,decoration={brace,amplitude=2mm}]
  ($(d1.east)+(1mm,0)$) -- ($(d7.south)+(0,-1mm)$) ;

\node [draw=none,anchor=east] at ($(c1)!0.5!(c7) + (-4mm,0)$)
 {\phantom{10}5\,000\,000--\\100\,000\,000\phantom{--}\\clients\\monitored}
;
\node [draw=none,anchor=south west] at ($(t1)!0.5!(t4) + (3mm,3mm)$)
 {Blocked sites\\(100--10\,000 servers)}
;
\node [draw=none,anchor=north west] at ($(d1)!0.5!(d7) + (0mm,-1mm)$)
 {Ignored sites\\($\sim$500\,000\,000 servers)}
;
\end{tikzpicture}%
\caption{Model topology for large-scale network censorship.}%
\label{f:cens.topology}%
\end{figure*}

\begin{figure*}[t!]%
\begin{tikzpicture}[every node/.style={align=center,draw}]
\matrix (a) [draw=none,row sep=2mm,nodes={anchor=east}]
{
  \node (a1) {Targeted client $1$} ;\\
  \foreach \y in {-1.5mm,0,1.5mm}
    \fill (a1.center |- 0,\y) circle (0.5mm) ;\\
  \node (a2) {Targeted client $M$} ;\\
  \node (a3) {Ignored client $1$} ;\\
  \foreach \y in {-1.5mm,0,1.5mm}
    \fill (a1.center |- 0,\y) circle (0.5mm) ;\\
  \node (a4) {Ignored client $N$} ;\\
};
\node [draw=none,below=1mm] at (a4.south) {\itshape 5--100
  clients\\\itshape total} ;
\matrix[draw=none,matrix anchor=b1.west,column sep=3mm,row sep=2mm]
  at ($(a2.east)!0.5!(a3.east) + (5mm,0)$)
{
&
&
&[10mm]
&
  \node (t1) [shape=circle] at (-6mm,6mm) {};
  \node (t2) [shape=circle,draw=none] at (-2mm,2mm) {};
  \node (t3) [shape=circle,draw=none] at (2mm,-2mm) {};
  \node (t4) [shape=circle] at (6mm,-6mm) {};
  \foreach \xy in {(-3mm,3mm),(-1mm,1mm),(1mm,-1mm),(3mm,-3mm)}
    \fill \xy circle (0.75mm) ;
\\
  \coordinate (b1) {};
&
  \node (b2) [draw=none,fill=gray,shape=circle,outer sep=0pt] {};
&
  \node (eve) {Eavesdropping\\router};
&
  \node (cloud) [cloud,aspect=2] {\strut};
&
\\
&
&
&
&
  \node (d1) [shape=circle] at (6mm,6mm) {};
  \node (d2) [shape=circle,draw=none] at (4mm,4mm) {};
  \node (d3) [shape=circle,draw=none] at (2mm,2mm) {};
  \node (d4) [shape=circle,draw=none] at (0mm,0mm) {};
  \node (d5) [shape=circle,draw=none] at (-2mm,-2mm) {};
  \node (d6) [shape=circle,draw=none] at (-4mm,-4mm) {};
  \node (d7) [shape=circle] at (-6mm,-6mm) {};
  \foreach \xy in {(-3mm,-3mm),(-1mm,-1mm),(1mm,1mm),(3mm,3mm)}
    \fill \xy circle (0.75mm) ;
\\
};
\node (b2l) [draw=none,anchor=base west,inner sep=0pt]
  at ($(b2.north)+(2mm,8mm)$) {\textit{0--2 hops}};
\draw (b2l.south west) edge ($(b2.north)+(0,0.5mm)$) ;

\foreach \n in {a1,a2,a3,a4} \draw (\n.east) edge (b1) ;
\draw ($(a1.east)!0.5!(a2.east)$) edge (b1)
      ($(a3.east)!0.5!(a4.east)$) edge (b1) ;

\draw (b1) edge (b2)
      (b2) edge (eve)
      (eve) edge (cloud) ;
\foreach \n in {t1,t2,t3,t4} \draw (cloud) edge (\n) ;
\foreach \n in {d1,d2,d3,d4,d5,d6,d7} \draw (cloud) edge (\n);
\draw [decorate,decoration={brace,amplitude=2mm}]
  ($(t1.north)+(0,1mm)$) -- ($(t4.east)+(1mm,0)$) ;
\draw [decorate,decoration={brace,amplitude=2mm}]
  ($(d1.east)+(1mm,0)$) -- ($(d7.south)+(0,-1mm)$) ;
\node [draw=none,anchor=south west] at ($(t1)!0.5!(t4) + (3mm,3mm)$)
 {Sites of interest\\(1000--100\,000 \emph{pages})}
;
\node [draw=none,anchor=north west] at ($(d1)!0.5!(d7) + (1mm,-1mm)$)
 {Ignored sites\\($\sim$10\,billion \emph{pages})}
;
\end{tikzpicture}%
\caption{Our proposed realistic topology for fingerprinting attacks.}%
\label{f:real.topology}%
\end{figure*}

Nearly all of the existing research on website fingerprinting has been
carried out in what we might call a laboratory environment.
Specifically, regardless of the attacker's goals (within-site,
identify-site, identify-user), experiments have been conducted using
roughly the network topology shown in figure~\ref{f:lab.topology}.
This topology is static; it does not change over the course of the
experiment.  There is only one traffic source, the targeted client,
and it is directly connected to the eavesdropping router.  The
unlinkability relay(s) are all on the far side of this router.  The
set of HTTP requests that the client might make is small and fixed.
For identify-site attacks, most of the literature attempts to
discriminate the front pages of a few hundred servers.  For
within-site attacks, there is only one server, and a similar number of
pages on that server are considered.  (Identify-user attacks have
historically been at larger scale than state-tracing attacks.)

Website fingerprinting is often discussed in the same breath as
large-scale surveillance and/or censorship of the global network, so
it is useful to contrast the usual assumed topology for that, shown in
figure~\ref{f:cens.topology}.  Note the enormously larger scale.
Country-scale “filtering” of the Internet must be enforced against
millions of clients, and must consider \emph{all} servers worldwide as
potential sources of undesirable material.  The number of servers that
actually do carry undesirable material is much smaller, and only some
of those come to the notice of the censor, but it is still a long
list.  Thus, country-scale censors are limited to lightweight
strategies for real-time blockade, such as IP address and keyword
blacklists on backbone routers.  More sophisticated techniques, such
as active probes for circumvention tools, are applied only when some
lightweight detection strategy triggers for a particular address, and
may take hours to react.~\cite{aase2012whiskey}~\needcite{active probe
  Tor thingy}

Fingerprinting is intrinsically heavier-weight than may be practical
on a backbone router.  General-purpose machine learning algorithms, as
used by most of the literature, have not been designed to run under
realtime constraints; even deliberately low-cost techniques,
e.g.\ that suggested in passing in~\cite{weinberg2012stegotorus},
require TCP stream reconstruction.  We suggest that a more plausible
location for a fingerprinting eavesdropper is a network hub near, but
not necessarily adjacent to, the target.  Unlike backbone routers,
these devices tend to be over-provisioned for the amount of traffic
they carry, so they have the headroom to run more elaborate analyses.
Also unlike backbone routers, they are built on general-purpose,
familiar computing environments (such as Linux or NetBSD) and their
management interfaces are more likely to be remotely exploitable to
install malware.\needcite{???}  However, network hubs, even at the
very edge of the network, are installed for the use of an entire
household (1--10 people) or office (10--50 people), or as a public
amenity, say at a café (20--100 people at any given time).  Even if
there is only one client \emph{computer} ever connected via the
eavesdropping router, that computer is likely to generate HTTP traffic
which is uninformative regarding the target \emph{human}: automatic
software updates, for instance, are carried out over HTTP nowadays.
If the eavesdropper is more than one hop from the target, their
situation is even worse: each IP address sending them traffic could
conceal dozens of clients behind a NAT gateway.  (NAT can be
detected~\cite{elie2005timestamp,krmivcek2009netflow} but that doesn't
help the adversary decide which packet streams to pay attention to.)

This brings us to the topology in figure~\ref{f:real.topology}.


The largest study to date~\cite{sun2002statistical} considered a
universe of the 20,000 most frequently contacted servers as recorded
by a departmental Web proxy.  Contrast the 670,000,000 distinct
websites reported in June 2013 by
Netcraft\footnote{\url{http://news.netcraft.com/archives/2013/06/06/june-2013-web-server-survey-3.html}}
or even the 100,000 sites which are sufficiently popular to have
stable Alexa
rankings\footnote{\url{http://www.alexa.com/faqs/?p=134}}.

Furthermore, only studies focusing on the contents of a \emph{single}
server investigate pages other than the “front page” of each server,
Front pages, which express the site's corporate identity, are
plausibly more different from each other than internal pages, which
are often a stock set of “chrome” wrapped around a blob of text.


\subsection{The Adversary's Goals}

To develop more realistic threat models for fingerprinting, we must
first consider the adversary's goals.  \todo{one of the early papers
did discuss this a bit}
