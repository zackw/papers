\documentclass{zarticle}
\usepackage{tikz}
\usetikzlibrary{shapes.symbols,shapes.geometric,decorations.pathreplacing,calc}
\addbibresource{shared.bib}
\definecolor{todocolor}{RGB}{127,0,0}
\def\todo#1{{\color{todocolor}\bfseries [#1]}}
\def\needcite#1{\todo{cite: #1}}
\begin{document}

\title{The Practicality of Website Fingerprinting at Global Scale}
\author{Zachary Weinberg}
\date{\today}
\maketitle

\begin{abstract}
\textit{Website fingerprinting} is a traffic-analytic attack which
aims to determine which websites, or perhaps even which specific
pages, a targeted individual is browsing, despite their use of an
“anonymity” service to conceal the ultimate destination of their
traffic.  A fingerprinting attacker loads each site of interest
themselves (using the same anonymity service), measures the network
traffic profile of each page load, and trains a machine classifier to
map traffic profiles back to sites.

To date, in the public literature, fingerprinting attacks have only
been attempted under laboratory conditions: only one targeted client,
no “background” network activity, and a tiny set of potential
destination pages, almost always the front pages of the “top
100--10,000 websites” taken from a list published by the Alexa
Corporation.  As such, despite plausible levels of success for a
laboratory test (classifier accuracies exceeding 90\% have been
claimed), it is unclear whether the attack is feasible “in the wild,”
facing orders of magnitude more servers (let alone \emph{pages}),
dozens to thousands of targeted clients (depending on the threat
model) with configurations not under the attacker's control, and
nontrivial levels of background traffic.

In this paper we compare the Alexa list to several other public
sources of big lists of URLs, \todo{summarize conclusions here}.
\end{abstract}

\section{Introduction}

\emph{Traffic analysis} is a class of techniques for extracting
information from a secure channel, using whatever data is still
observable despite encryption.  It is a staple of signals
intelligence, practiced for over a century, and historically quite
successful in the field.~\cite{kahn1967codebreakers} Most secure
channel protocols do not conceal metadata such as the source,
destination, establishment time, and duration of the connection, and
the protocol in use.  To illustrate the sensitivity of this
information, \textcite{felten2013decl} observes that the mere fact of
a call placed to certain special-purpose telephone numbers reveals
“basic and often sensitive” information about the caller.  There are
dedicated support hotlines for survivors of domestic violence, people
considering suicide, and sufferers from various forms of addiction;
there are also lines dedicated to anonymously reporting government
misconduct, tax fraud, illegal firearms, and other crimes.
\citeauthor{felten2013decl} also points out that if phone call records
are analyzed over time, they reveal the “social graph” and can even
indicate the nature of personal connections; for instance, regular
late-night phone calls suggest a long-distance intimate relationship.

\subsection{Unlinkability}

Because it is easy to extract information from the source and
destination of encrypted communications, protocols have been developed
specifically to conceal this information.  In addition to all the usual
properties of a secure channel, an \emph{unlinkable} channel protocol
seeks to ensure that no eavesdropper at a single point in the network
can learn \emph{both endpoints} of the channel.  That is, no
eavesdropper can tell that Alice is talking to Bob over an unlinkable
channel, but eavesdroppers can still tell that Alice and Bob are
talking to \emph{someone}.  (This definition is due to
\textcite{pfitzmann2010terminology}.  Whether or not active attackers
are considered depends on the writer.)

The standard technique for providing unlinkability, invented by
\textcite{chaum1981mix}, is to pass each message through one or more
“relays,” (also known as “mixers”), layering encryption so that each
relay only knows the preceding and next link in the chain. One relay
is enough against most eavesdroppers, but provides no protection if
the relay itself is subverted.  Operators of popular single-hop “proxy
services” or “anonymous remailers” may, and indeed have, come under
coercion to expose publishers of controversial
content.~\cite{newman1996church, singel2007hushmail,
  ackerman2013lavabit} Similarly, a two-hop chain is vulnerable to a
pair of cooperating malicious relays.  In principle, a three-hop chain
is secure unless the adversary controls a significant fraction of the
network, and more hops only add overhead.~\cite{wright2002analysis,
  wright2003defending}

Chaum's original design was intended for email. It imposed significant
delays at each relay, in order to conceal the transmission time of
each message.  This delay is unacceptable for interactive applications
(such as online chat and Web browsing), so the current generation of
“low latency” unlinkable protocols, such as
Tor~\cite{dingledine2004tor} and I2P~\cite{i2p.undated.i2p},
deliberately omit it.  This opens an avenue for “intersection” attacks
by adversaries who \emph{do} control a significant fraction of the
network, or who can observe traffic at many points in the network;
these are formally out of scope, but have still received a good deal
of attention in the literature, e.g.~\cite{danezis2003statistical,
  danezis2004continuous, danezis2005statistical, danezis2007twosided,
  murdoch2007sampled, shmatikov2006timing, overlier2006locating}
Defenses against these attacks are still being researched, although a
few simple tactics have been deployed, such as “guard
nodes”~\cite{overlier2006locating}: the attacks are most effective if
a malicious node is directly connected to a client, so by always
beginning one's chains with the same node for an extended period, one
avoids having any of one's traffic be exposed to a malicious node
(assuming that the initial choice is sound).

\subsection{Website Fingerprinting}

\input http.tikz

The standard mathematical definition of “confidentiality” assumes that
all messages are the same length, even though in practice, this is not
true.  This is not a minor loophole; as recently
demonstrated~\cite{duong2012crime,gluck2013breach}, compressing data
before encryption (thus causing each message's length to depend
directly on its contents) can allow an active attacker to recover the
plaintext.  Pure eavesdroppers can also learn something from the
length of a message.  For instance, if there is a server that will,
for anyone, upon request, produce one of a relatively small number of
static documents, then the eavesdropper can record all such documents'
lengths in advance, and from then on, determine which one is being
sent to a surveillance target by measuring the length of the
transmission.  This describes at least part of the behavior of nearly
all public Web servers.

HTTP exhibits characteristic network behavior, illustrated in
figure~\ref{f:packet-trace}, which is useful for traffic analysis.  As
described in more detail in that figure, an eavesdropper at the TCP
layer will see each party alternate between sending and receiving, and
the aggregate TCP payload size of each server-to-client burst is a
close approximation to the file size of the object requested.  To
enhance parallelism, browsers open anywhere between two and ten
simultaneous connections to the same site, allowing them to request
several items at once (e.g.\ all the images referenced by an HTML
page); each such connection behaves as described.

Some, but not all, unlinkable channel protocols disguise the
characteristic HTTP traffic pattern to some extent.  For instance, Tor
multiplexes all proxied TCP connections onto a small number of
long-lived “circuits,” hiding the simultaneous connections aspect of
HTTP.  The alternating send and receive bursts are still visible, but
it is harder to determine the exact size of each item.


\printbibliography
\clearpage
\appendix
\section{Figures}

\input threat.tikz

\end{document}
