\subsection{Overlap Between Lists}

\begin{table*}
\caption{Jaccard coefficients for list similarity, by URL}
\label{t:list-sim-url}
\centering\sffamily\input plots/list-sim-url.tik\relax
\end{table*}

\begin{table*}
\caption{Jaccard coefficients for list similarity, by hostname}
\label{t:list-sim-host}
\centering\sffamily\input plots/list-sim-host.tik\relax
\end{table*}

We begin our investigation by comparing the probe lists to each other,
using the Jaccard index of similarity: \hbox{$J=\frac{\lvert A\cap
  B\rvert}{\lvert A\cup B\rvert}$} for any two sets $A$ and $B$.  It
ranges from 0 (no overlap at all) to 1 (complete overlap).

Table~\ref{t:list-sim-url} shows the Jaccard indices for each pair of
lists, comparing full URLs.  It is evident that, although there is
some overlap (especially among the pinklists, in the upper left-hand
corner), very few full URLs appear in more than one list.  There is
more commonality if we look only at the hostnames, as shown in
Table~\ref{t:list-sim-host}.  The pinklists continue clearly to be
more similar to each other than to anything else.  The blacklists,
interestingly, continue not to have much in common with each other.
And, equally interestingly, all the other lists---regardless of
sampling criteria---have more in common with each other than they do
with most of the blacklists and pinklists.  This already suggests that
manually curated lists such as ONI's may not be digging deeply enough
into the “long tail” of special-interest websites.

While we can see that there are patterns of similarities,
Tables~\ref{t:list-sim-url} and~\ref{t:list-sim-host} do not reveal
\emph{what} some lists have in common with each other.  To discover
that, we must study the content of each page, which is the task of the
rest of this paper.
