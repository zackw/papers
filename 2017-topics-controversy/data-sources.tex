\section{Lists tested}\label{s:lists}

We studied 758,191 unique URLs drawn from 22 test lists
(shown in Table~\ref{t:list-sim-url}).  Only one was
created to be used as a probe list~\cite{oni.nd.testlists}, but
another 15 are (allegedly) actual blacklists used in specific
countries, and two more have algorithmic selection criteria that
should be positively correlated with censorship.  The remaining four
are control groups.  One should be \emph{negatively} correlated with
censorship, and the others are neutral.

Due to the sheer size and diversity of the global Web, and the large
number of pages that are not discoverable by traversing the link
graph~\cite{biemann2013scalable}, any sample will inevitably miss
something.  We cannot hope to avoid this problem, but drawing our
sample from a wide variety of sources with diverse selection criteria
should mitigate it.

\subsection{Potentially censored}\label{s:lists-censored}

Pages from these lists should be more likely than average to be
censored somewhere.

\paragraph{Blacklists and pinklists}\label{s:lists-blacklists} These
documents purport to be (part of) actual lists of censored URLs in
some countries.  Most are one-time snapshots; some are continuously
updated.  They must be interpreted cautiously.  For instance, the
leaked “BlueCoat” logs for Syria~\cite{chaabane.2014.syria} list
only URLs that someone tried to load; there is no way of knowing
whether other URLs are also blocked, and one must guess whether entire
sites are blocked or just specific pages.

This study includes 15 lists from 12 countries, for a total of 331,362
URLs.  Eight of them include overwhelmingly more pornography than
anything else; we will refer to these as \emph{pinklists} below.  (All
eight do include some non-pornographic sites, even though six of them
are from countries where the ostensible official policy is \emph{only}
to block pornography.)  The other seven do not have this emphasis, and
we will refer to them as \emph{blacklists} below.

\paragraph{OpenNet Initiative}\label{s:lists-oni}
ONI is an international research institute devoted to the study of
Internet censorship and surveillance.  They publish a hand-curated
probe list of 12,107 URLs discussing sensitive
topics~\cite{oni.nd.testlists}.  The principle is that these are more
likely to be censored than average, not that they necessarily
\emph{are} censored somewhere. We take this list as representative of
the probe lists used by researchers in this field.  1,227 of the URLs
are labeled as globally relevant, the rest as relevant to one or more
specific countries.

Hand-curated lists will inevitably reflect the concerns of their
compilers.  The ONI list, for instance, has more “freedom of expression
and media freedom” sites on the list than anything else.

\paragraph{Herdict}\label{s:lists-herdict}
Herdict~\cite{berkman.nd.herdict} is a service which aggregates
worldwide reports that a website is inaccessible.  A list of all the
URLs ever reported can be downloaded from a central server; this comes
to 76,935 URLs in total.  The browser extension for making reports is
marketed as a censorship-reporting system, but they do not filter out
other kinds of site outage.  This list includes a great deal of junk,
such as hundreds of URLs referring to specific IP addresses that serve
Google's front page.

\paragraph{Controversial Wikipedia articles and their references}
\label{s:lists-wikipedia}
\Textcite{yasseri.2014.wikipedia} observe that controversy on
Wikipedia can be mechanically detected by analyzing the revision
history of each article. Specifically, if an article's history
includes many “mutual reverts,” where pairs of editors each roll back
the other's work, then the article is probably controversial.  (This
is a conservative measure; as they point out, Wikipedia's edit wars
can be much more subtle.)  They published lists of controversial
Wikipedia articles in 13 languages.  We augmented their lists with the
external links from each article.  This came to a total of 105,181
URLs.

\subsection{Controls}

These lists were selected to reflect the Web at large.

\paragraph{Pinboard}\label{s:lists-pinboard}
We expect pages on this list to be \emph{less} likely to be censored
than average. It is a personal bookmark list with 3,276 URLs,
consisting mostly of articles on graphic design, Web design, and
general computer programming, with the occasional online shopfront.

\paragraph{Alexa 25K}\label{s:lists-alexa}
Alexa Inc.\ claims that these are the 25,019 most popular websites
worldwide; their methodology is opaque, and we suspect it over-weights
the WEIRD (Western, Educated, Industrialized, Rich, and
Democratic~\cite{henrich2010weird}) population.  Sensitive content is
often only of interest to a narrow audience, and the popularity of
major global brands gives them some protection from censorship, so
sites on this list may also be less likely to be censored than
average.

\paragraph{Twitter}\label{s:lists-twitter}
Another angle on popularity, we use a small (less than 0.1\%) sample
of all the URLs shared on Twitter from March 17 through 24, 2014,
comprising 30,487 URLs shared by 27,731 user accounts.  Twitter was
chosen over other social networks because, at the time of the sample,
political advocacy and organization via Twitter was fashionable.

\paragraph{Common Crawl}\label{s:lists-ccrawl}
Finally, this is the closest available approximation to an unbiased
sample of the entire Web.  The Common Crawl Foundation continuously
operates a large-scale Web crawl and publishes the
results~\cite{commoncrawl}. Each crawl contains at least a billion
pages. We sampled 177,109 pages from the September 2015 crawl
uniformly at random.

\input eval-list-jaccard
