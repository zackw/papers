\subsection{Survival Analysis}\label{s:survival-desc}

\input diagrams/survival

Our data on the life cycle of websites is unavoidably incomplete.
Figure~\ref{f:survival-diagram} shows four hypothetical cases which
illustrate the problem.  In no case do we know when a page was
created, only when it first came to the attention of the Wayback
Machine.  If a page survives to the present (A, C), we do not know how
much longer it will continue to exist.  If it was abandoned (B, D), we
only know that this happened within an interval between two
observations.  If a page appears on a censorship blacklist (C, D), we
know when this happened (blue dot) but we can only guess at how long
the page was censored (blue shaded area).

\emph{Survival analysis}~\cite{KM58} is a set of statistical
techniques originally developed for predicting the expected lifespan
of patients with terminal illnesses.  Because medical studies often
suffer from exactly the same kinds of gaps in their data---one doesn't
usually know how long a tumor was present before it was diagnosed, for
instance---survival analysis is prepared to deal with them.  However,
to use these techniques we must define what it means for a page to
“die.”  Clearly, if the site is shut down or turns into a parked
domain, that should qualify.  Less obviously, we also count topic
change as “death.”  This is because, after a censored page has changed
topic, it no longer provides the same kind of sensitive material that
it used to, so it can no longer be considered “fresh.”  For example,
the blog \nolinkurl{http://amazighroots.blogspot.com} was taken over
by spammers in 2014, and its apparent topic changed from “news and
politics” to “food.”  Probing such pages longer reveals whether the
censor cares about that kind of sensitive material.  It may instead
reveal how diligent the censor is about updating their blacklist, but
this was not the original goal.

Unlike medical patients, webpages may be “dead” only temporarily, due
to server crashes, vandalism, the owners forgetting to renew the
domain registration, and so on.  Stock survival analysis does not
allow for this possibility.  To handle it, we calculated every
survival curve two ways: first, assuming that revivals never happen
(once a site “dies” it is treated as staying that way, even if we have
evidence to the contrary) and second, allowing sites to revive even
after an arbitrarily long hiatus.  The first approach gives an
underestimate of survival times, the second, an overestimate.
Figure~\ref{fig:km-base} shows, over all pages we monitored, that on
average, the differences between both approaches are relatively small;
and that our error ranges are small.

\begin{figure}
\centering\sffamily\input plots/km-basic.tik\relax
\caption{Comparison of the two approaches to page revival.\\
Shading shows confidence intervals.}
\label{fig:km-base}
\end{figure}
